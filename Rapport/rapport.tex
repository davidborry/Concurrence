% !TEX TS-program = pdflatex
% !TEX encoding = UTF-8 Unicode

% This is a simple template for a LaTeX document using the "article" class.
% See "book", "report", "letter" for other types of document.

\documentclass[11pt]{article} % use larger type; default would be 10pt

\usepackage[utf8]{inputenc} % set input encoding (not needed with XeLaTeX)

%%% Examples of Article customizations
% These packages are optional, depending whether you want the features they provide.
% See the LaTeX Companion or other references for full information.

%%% PAGE DIMENSIONS
\usepackage{geometry} % to change the page dimensions
\geometry{a4paper} % or letterpaper (US) or a5paper or....
% \geometry{margin=2in} % for example, change the margins to 2 inches all round
% \geometry{landscape} % set up the page for landscape
%   read geometry.pdf for detailed page layout information

\usepackage{graphicx} % support the \includegraphics command and options

% \usepackage[parfill]{parskip} % Activate to begin paragraphs with an empty line rather than an indent

%%% PACKAGES
\usepackage{booktabs} % for much better looking tables
\usepackage{array} % for better arrays (eg matrices) in maths
\usepackage{paralist} % very flexible & customisable lists (eg. enumerate/itemize, etc.)
\usepackage{verbatim} % adds environment for commenting out blocks of text & for better verbatim
\usepackage{subfig} % make it possible to include more than one captioned figure/table in a single float
% These packages are all incorporated in the memoir class to one degree or another...

%%% HEADERS & FOOTERS
\usepackage{fancyhdr} % This should be set AFTER setting up the page geometry
\pagestyle{fancy} % options: empty , plain , fancy
\renewcommand{\headrulewidth}{0pt} % customise the layout...
\lhead{}\chead{}\rhead{}
\lfoot{}\cfoot{\thepage}\rfoot{}

%%% SECTION TITLE APPEARANCE
\usepackage{sectsty}
\allsectionsfont{\sffamily\mdseries\upshape} % (See the fntguide.pdf for font help)
% (This matches ConTeXt defaults)

%%% ToC (table of contents) APPEARANCE
\usepackage[nottoc,notlof,notlot]{tocbibind} % Put the bibliography in the ToC
\usepackage[titles,subfigure]{tocloft} % Alter the style of the Table of Contents
\renewcommand{\cftsecfont}{\rmfamily\mdseries\upshape}
\renewcommand{\cftsecpagefont}{\rmfamily\mdseries\upshape} % No bold!
\usepackage{hyperref}
%%% END Article customizations

%%% The "real" document content comes below...

\title{Projet programmation concurrente \\
	Polytech'Nice Sophia - SI4 G1}
\author{David BORRY\\
		Thomas GILLOT}
%\date{} % Activate to display a given date or no date (if empty),
         % otherwise the current date is printed 

\begin{document}
\maketitle

Nous déclarons sur l'honneur que ce rapport et l'application qu'il décrit sont le fruit de notre propre travail, basé sur notre expérience scolaire et personnelle sur ces dernières années et que nous n'avons ni contrefait, ni falsifié, ni copié partiellement sur l'oeuvre d'autres binômes ou sur internet. Nous sommes conscients que le plagiat est considéré comme une faute grave pouvant être sévèrement sanctionnée. 

\newpage

\section{Introduction}
Ce rapport a pour sujet la conception et le développement d'une simulation de déplacement d'une foule d'individus dans un environnement comprenant plusieurs obstacles, basée sur les notions vues en cours et en TD de programmation concurrente.  \\
L'application a été écrite en C++ et utilise la bibliothèque POSIX du langage C.
Pour la représentation graphique, c'est la bibliothèque SFML qui est utilisée.
Les tests unitaires sont quant à eux réalisés avec Google Test. \\
Il n'est pas nécessaire d'avoir ces bibliothèques installées au préalable et l'application peut être compilée et éxécutée sans la partie graphique. \\
Le rapport a pour but d'expliquer le fonctionnement de l'application et les décisions prises pour la conception des principaux algorithmes,  analyser et comparer le fonctionnement des threads POSIX par rapport à Java ainsi que les performances en fonction des différentes configurations possibles



\tableofcontents

\newpage
\section{Conception}
La simulation a un objectif clair: faire converger une quantité plus ou moins grande de personnes vers une unique destination et faire en sorte qu'ils finissent tous par l'atteindre. Toutefois, elle a trois manières différentes de remplir cet objectif. La première consiste à déplacer successivement chaque personne d'une unité jusqu'à ce qu'elles aient toute atteint leur destination. La seconde permet de gérer de la même manières ces personnes, mais en fonction de leur position l'une des quatre régions de tailles égales constituant le monde. Une thread est associé à chaque région. Pour la troisième technique, on crée une thread pour chaque personne à déplacer. 

\subsection{Fonctionnement général}
\subsubsection{Base du programme}
Trois scénarios bien différents peuvent donc s'appliquer, chacun ayant ses particularités en performances et en algorithmique.
Il est néanmoins important qu'elles partagent la plupart des variables et des algorithmes du programme afin qu'il soient plus simple à maintenir et à faire évoluer. Quelque soit le scénario, on a donc toujours : 
\begin{itemize}
	
	\item L'ensemble des murs et des personnes à déplacer sur le terrain 
	\item Une destination fixée à une extrémité du terrain.
	\item Une carte du terrain constamment mise à jour où sont représentés les obstacles qu'une personne peut rencontrer (murs ou 			autres personnes).
	\item Les algorithmes permettant de créer et de déplacer d'une unité une personne en direction d'une destination donnée.
\end{itemize}

\subsubsection{Déplacement une personne}
Le bon fonctionnement du programme repose en grande partie sur l'algorithme de déplacement. Le programme ne se terminant que lorsqu'il n'y a plus personne à déplacer, il faut un algorithme rapide et solide prenant en compte la gestion des obstacles. Un algorithme de pathfinding pour éviter les obstacles comme \textit{Dijkstra} ou \textit{A*} serait inneficace car peu optimal avec un grand nombre de personnes à gérer et/ou une grande carte. \\
En revanche, un algorithme de type \href{https://gamedevelopment.tutsplus.com/tutorials/understanding-steering-behaviors-seek--gamedev-849}{\textit{steering behaviour}} principalement basé sur les notions de vecteurs et de distances est bien plus approprié.
L'entité à déplacer ne pourra pas éviter les obstacles de manière intelligente s'il y en a, c'est pourquoi il est possible qu'une destination fixée soit impossible à atteindre pour certaines entités (si la destination est derrière un mur et que la seule entrée pour y accéder se situe à l'autre bout du terrain, par exemple). Le choix de la destination est donc également essentiel pour que la simulation s'éxécute correctement.

\newpage

L'algorithme choisi pour la simulation fonctionne donc de cette manière : \\

Pour un \textbf{terrain}, une \textbf{entité}  et une \textbf{destination} données, si l'entité n'a pas atteint la destination :
\begin{itemize}
	\item Stocker dans un ensemble les vecteurs de directions des bordures de l'entité par rapport à la destination.
	\item tant que l'ensemble des vecteurs n'est pas vide:
	\begin{itemize}
		\item Retirer le vecteur ayant la plus petite longueur de l'ensemble
		\item Essayer de déplacer d'une case l'entité en suivant ce vecteur
		\item Si l'entité a bougé, arrêter.
		\item Sinon, continuer.
	\end{itemize}
	
\end{itemize}

\subsection{Gestion des threads}
Si les scénarios partagent donc la plupart des ressources du programmes, ils les utilisent tous d'une manière différente. 
Pour l'option \textbf{-t1} où la simulation est gérée par la thread principale, le fonctionnement est assez simple : Tant que toutes les personnes n'ont pas atteint la destination, répéter pour chacune d'entre elles l'algorithme de déplacement décript précédemment. On doit donc mettre à jour le monde et l'ensemble des personnes un certain nombre de fois, ce qui peut faire penser au design d'une boucle de jeu. C'est le but : de cette manière, on peut facilement intégrer la représentation graphique à la simulation en redessinant le terrain à chaque mise à jour. \\
Pour l'option \textbf{-t2}, le fonctionnement est assez semblable, mais on divise le terrain en 4 parties à gérer pour chaque thread créée. Les threads ont toujours une boucle devant déplacer des entités, mais seulement celles se trouvant dans la région associée. \\
L'option \textbf{-t3} utilise toujours une boucle, mais elle ne gère qu'une entité par thread. La boucle continue tant que l'entité n'a pas atteint sa destination.

\subsubsection{Les threads POSIX}

\subsubsection{Intégrer les threads à la simulation}
Pour les deux configurations multithread, on doit faire en sorte que chaque thread puisse s'éxécuter entièrement, sans que le programme ne se termine avant. C'est rendu possible grâce aux fonctions \textbf{pthread\_create} et \textbf{pthread\_join} de la bibliothèque POSIX. Les algorithmes de gestion des threads sont donc très semblables pour les simulations t1 et t2. Ils fonctionnent de la manière suivante : 

\newpage

\textbf{t1} :
\begin{itemize}
\item Pour chacune des quatre régions du terrain :
	\begin{itemize}
		\item Créer une nouvelle thread
		\item Associer à cette thread la gestion des entités situées dans la région actuelle
	\end{itemize}

\item Attendre la terminaison de chacune des threads précédemment créées.

\end{itemize} 
 
\textbf{t2} :
\begin{itemize}
\item Pour chaque entité sur le terrain :
	\begin{itemize}
		\item Créer une nouvelle thread
		\item Associer à cette thread la gestion de l'entité actuelle
	\end{itemize}

\item Attendre la terminaison de chacune des threads précédemment créées.

\end{itemize}

\section{Choix de développement}
En dehors de la gestion des threads et des options entrées par l'utilisateur, le programme est principalement développé en C++ et tire parti de la programmation orientée objet, du polymorphisme et de la surcharge d'opérateurs. \\
La carte du monde est ainsi représentée par un tableau à deux dimensions contenant des cellules pouvant être \textbf{solides} (occupées par un obstacle) ou non. \\
Les vecteurs utilisés pour déplacer nos entités ont leur propre structure, \textbf{Vector2i} et peuvent être additionnés, soustraits, comparés par leur longueur et multipliés par un scalaire. Les régions du terrain à gérer pour le scénario \textbf{t1} ont également une structure \textbf{Rectangle} comprennant les points Nord-Ouest et Sud-Est de la région. \\
La classe \textbf{World} permet de gérer les déplacement d'une partie ou de la totalité des entités du terrain. Les entités à déplacer sont stockées dans une liste de pointeurs, et elles sont modélisées dans la plus grande classe de l'application.

\subsection{Modéliser une entité}
Il faut d'abord pouvoir définir ce qu'est une entité dans un monde représenté par un ensemble de cellules. 
Une \textbf{entité} est une forme rectangulaire composée de cellules, pouvant être solide ou non. Elle est définie par une position, une longueur, une largeur et un booléen définissant sa solidité. Elle peut également se déplacer dans toutes les directions et elle peut être détruite avant de réapparaître à sa position d'origine. \\
Une entité peut donc être une personne, un mur ou un trou. Les seules données qui changent à chaque type sont les dimensions et la solidité, c'est pourquoi il est inutile de créer une nouvelle classe fille pour chaque type. Il est préférable de stocker ces données dans un fichier (ici \textbf{DataTable}) afin de les utiliser simplement et rapidement dans le constructeur. Le reste des attributs et fonctions, en particulier celle de déplacement, ne changent pas.
 L'algorithme de déplacement cité précédemment est implémenté dans la fonction \textbf{update}. \\
Cette fonction n'est appelée que si l'entité à une destination à atteindre (\textbf{mTarget}). Grâce à la structure \textbf{Vector2i}, le calcul d'une direction à suivre est assez simple : On soustrait au vecteur destination un vecteur position donné. Pour l'entité à déplacer, cette position est basée sur les bordures, à savoir les cellules extérieures du rectangle. Ainsi, pour un humain occupant 4x4 cellules, 12 directions sont calculées. Ces directions sont ensuite stockées dans une structure de données \textbf{Priority Queue} où elles sont rangées dans l'ordre croissant. De cette manière, chaque retrait de la tête de file renverra la direction à la plus petite distance. \\

Quant au déplacement d'une case en lui-même, son implémentation est également basique : On vérifie qu'il n'y a pas d'obstacle et s'il y en a un, on peut annuler la vélocité verticale et/ou horizontale de l'entité en fonction de sa position. \\
On remarquera que la classe World dispose également d'une fonction \textbf{update}, dans laquelle on applique la fonction du même nom pour chacune des entités à déplacer. Comme il a été dit précédemment cela ressemble fortement à une boucle de jeu, facilitant ainsi l'intégration de la représentation graphique.

\subsection{Simulation et polymorphisme}
L'application en elle-même est lancée à travers la classe \textbf{Simulation}. On génère le monde et les entités dans le constructeur, et on lance la boucle dans la fonction \textbf{run}. Cette fois-ci, l'usage du polymorphisme est avantageux car cette fonction doit avoir un comportement bien différent pour chaque scénario. On crée donc des classes filles \textbf{MTSim1} et \textbf{MTSim2} respectivement associées aux configurations \textbf{-t1} et \textbf{-t2}. La fonction run est alors redéfinie pour pouvoir gérer la création des threads et l'attente de leur terminaison. Comme les threads POSIX ne peuvent pas gérer des fonctions membres directement, on recrée dans ces classes des fonctions \textbf{statiques} pour gérer les déplacement dans le monde. 

\section{Tester l'application}
\subsection{Tests unitaires}
\subsection{Performances}

\section{Conclusion}

\end{document}
